\documentclass[11pt]{article}

% -------------------------------------------------------------------------------- %

% *** LANGUAGE ***
\usepackage[english,italian]{babel}
\usepackage[utf8]{inputenc}

% *** PAGE LAYOUT ***
\usepackage{geometry}     
\geometry{a4paper,tmargin=2cm,bmargin=2cm,lmargin=2cm,rmargin=2cm}
\usepackage{parskip}  
\usepackage{enumerate}  
\setlength{\parindent}{0pt}


% *** DEFAULT FONT SELECTION ***
\usepackage[T1]{fontenc}
\usepackage{lmodern}
\renewcommand*\familydefault{\sfdefault} %	Only if the base font of the document is to be sans serif

% *** GRAPHICS *** 
\usepackage{graphicx}
\usepackage{color}
\usepackage{float}

% *** AMS MATH ***
\usepackage{amsmath, amsfonts, amssymb, bm}

% *** ADDITIONAL MATH ***
\usepackage{empheq}
\usepackage{systeme}
\usepackage{mathtools}

\newlength{\cellwidth}
\settowidth{\cellwidth}{$\displaystyle -2\,\omega_g\, r_g$}

% *** TABLES ***
%\usepackage{booktabs,topcapt}
\usepackage{multirow}


\newcommand\scalemath[2]{\scalebox{#1}{\mbox{\ensuremath{\displaystyle #2}}}}

\usepackage{subcaption}
\usepackage{siunitx}


\usepackage[title]{appendix}


% *** OTHER PACKAGES ***
\usepackage{lipsum}  
\usepackage{url}
\usepackage{fancyhdr}

\usepackage{hyperref}
\pagestyle{fancy}
\setlength{\headheight}{15pt}
\usepackage{titlesec}
\titleformat{\section}
{\normalfont\sffamily\large\bfseries}
{\thesection}{1em}{}
\titleformat{\subsection}
{\normalfont\sffamily\bfseries}
{\thesubsection}{1em}{}
\renewcommand{\thesubsection}{\thesection.\alph{subsection})}

\lhead{Laurea Magistrale in Ingegneria Meccatronica}
\rhead{A.A. 2022-23}


% -------------------------------------------------------------------------------- %

\title{\vspace{-0.75cm} \Large \textbf{Progettazione di Sistemi di Controllo}\\[0.2cm] \Large \textbf{Controllo Ottimo del pendolo di Furuta} 
}
\author{Marco Pallaro \\[0.2cm] 2055621 } 
\date{}

% -------------------------------------------------------------------------------- %

%\usepackage{fancyhdr}
%\lhead{Nome Cognome} 
%\chead{} 
%\rhead{LABORATORIO \#}
%\lfoot{}
%\cfoot{ \thepage/\pageref{LastPage} } 
%\rfoot{} 
%\renewcommand{\headrulewidth}{1pt}
%\renewcommand{\footrulewidth}{0pt}


% -------------------------------------------------------------------------------- %


\begin{document}

\maketitle				%	print title




\section*{Abstract - To Be Revised}

Il fine dell'attività di laboratorio è il design e lo studio del controllo ottimo del pendolo di Furuta.
In particolare, il controllo ha come obiettivo la stabilizzazione del pendolo nella sua posizione di equilibrio instabile, ovvero la posizione verticale.
Di conseguenza, il primo passo è quello di decidere, attraverso diverse prove, quali guadagni o pesi usare nella funzione costo del controllo LQR.
Successivamente, ottenuto un controllo funzionante, si passa alla discretizzazione di tale controllore, in modo da avvicinarsi alla realtà del problema.
Di conseguenza si aumenta la complessità del sistema provando ad imporre una traiettoria di inseguimento alla base del pendolo.
Infine, si ricerca un controllore per lo \textit{swing-up}: si studia un metodo per portare (senza aiuti esterni) il pendolo da una posizione di riposo, e dunque dalla posizione di equilibrio stabile, alla posizione di equilibrio instabile studiata finora, con il pendolo verticale. 


\section{Introduzione?}

\subsection{Descrizione del sistema}


\subsubsection*{Modello matematico del pendolo}

\begin{figure}[H]
	\centering
	\includegraphics[width=.3\columnwidth]{example-image-a}
    \includegraphics[width=.3\columnwidth]{example-image-a}
	\caption{Evoluzione forzata del sistema in corrispondenza degli ingressi $u_1(t)$ e $u_2(t)$. Si considerano condizioni iniziali nulle}
	\label{fig:11}
\end{figure}

Per ricavare le equazioni matematiche del sistema si può utilizzare l'approccio lagrangiano. 
\begin{equation}
	L = T - U, \qquad \sum_{i}^{} W_i = \dfrac{dL}{dt}
\end{equation}
dove $L$ viene definito lagrangiano: con $T$ ci si riferisce all'energia cinetica e con $U$ ci si riferisce all'energia potenziale del sistema. Con $\sum W$ si considerano tutte le potenze agenti sul sistema causate solamente da forze esterne a questo. 


Rispetto al caso di studio, innanzitutto è possibile riconoscere due diversi sottosistemi, la base con il motore e il modulo con il pendolo. Si definiscono le seguenti notazioni: $\theta$, $\alpha$ sono rispettivamente le coordinate angolari nel piano orizzontale e del piano verticale. $l_r$ e $l_p$ sono rispettivamente la lunghezza del braccio orizzontale e del pendolo (dal punto di ancoraggio al centro di massa del pendolo). $J_r$ e $J_p$ sono le inerzie della base e del pendolo, $m_p$ la massa del pendolo. Infine $r_{cm}$ è il vettore che congiunge l'origine del sistema al centro di massa del pendolo.

Passando all'analisi energetica, per quanto riguarda la base, è presente l'energia cinetica collegata all'inerzia del motore:
\begin{equation}
    T_{base} = \dfrac{1}{2} \, J_r \, \dot{\theta}^2
\end{equation}

Per il pendolo, oltre al termine cinetico, è presente anche il contributo di energia potenziale, perciò:
\begin{equation}
	\begin{aligned}
		&T_{pendolo} = \dfrac{1}{2} \, J_p \, \dot{\alpha}^2 + \dfrac{1}{2} \, m_p \, \dot{r}_{cm}^2\\
		&U_{pendolo} = m_p \, g\, l_p\, \cos\alpha 
	\end{aligned}
\end{equation}

Il calcolo di $\dot{r}_{cm}^2$ non è immediato in quanto si sta considerando il vettore $r_{cm}$, prima da derivare e successivamente da elevare al quadrato. In particolare: 
\begin{equation}
	r_{cm} = 
	\left\{
	\begin{aligned}
		&l_r \, cos\theta - l_p\,sin\alpha\,sin\theta \\
		&l_r \, sin\theta + l_p\,sin\alpha\,cos\theta \\
		&l_p\,cos\alpha
	\end{aligned}
	\right.
	\qquad\rightarrow\qquad
	\dot{r}_{cm}^2 = l_r^2\dot{\theta}^2 + l_p^2\,(\dot{\alpha}^2 + \dot{\theta}^2\,sin^2\alpha) + 2\,l_rl_p\,\dot{\theta}^2\dot{\alpha}^2\,cos\alpha
\end{equation}

Sommando e sostituendo si trova il lagrangiano $L = T - U$:
\begin{equation}
	L = \dfrac{1}{2} \, J_r \, \dot{\theta}^2 + \dfrac{1}{2} \, J_p \, \dot{\alpha}^2 + \dfrac{1}{2} \, m_p \, l_r^2\, \dot{\theta}^2 + \dfrac{1}{2} \, m_p \, l_p^2\, \dot{\alpha}^2 + \dfrac{1}{2} \, m_p \, l_p^2\, \dot{\theta}^2\, sin^2\alpha + m_p\,l_r\,l_p\,\dot{\theta}\,\dot{\alpha}\,cos\alpha - m_p\,g\,l_p\,cos\alpha
\end{equation}



Poiché nel sistema sono presenti due coordinate (gli angoli $\theta$ e $\alpha$) sono necessarie due equazioni per descrivere il comportamento del sistema:
\begin{equation}
	\begin{aligned}
		&\dfrac{d}{dt}\dfrac{\partial L}{\partial \dot{\theta}} - \dfrac{\partial L}{\partial \theta} = C_m - b_r \, \dot{\theta} \\[2ex]
		&\dfrac{d}{dt}\dfrac{\partial L}{\partial \dot{\alpha}} - \dfrac{\partial L}{\partial \alpha} = - b_p \, \dot{\alpha}
	\end{aligned}
\end{equation}
Si deriva dunque il lagrangiano rispetto $\theta$ e $\alpha$; si ottengono:
\begin{equation}
	\begin{aligned}
		&J_r\, \ddot{\theta} + m_p\,l_r^2\, \ddot{\theta} + m_p\,l_p^2\,sin2\alpha\,\dot{\alpha} + m_p\,l_p^2\,sin^2\alpha\,\ddot{\theta}+m_p\,l_r\,l_p\,cos\alpha\,\ddot{\alpha} - m_p\,l_r\,l_p\,sin\alpha\,\dot{\alpha}^2 = C_m - b_r\, \dot{\theta}\\[1ex]
		&J_p\,\ddot{\alpha} + m_p\,l_p^2\,\ddot{\alpha} + m_p\,l_r\,l_p\,\ddot{\theta}\,cos\alpha - m_p\,l_p^2\,\dot{\theta}^2\,sin\alpha\cos\alpha - m_p\,g\,l_p\,sin\alpha= - b_p \, \dot{\beta}
	\end{aligned}
\end{equation}




\subsubsection*{Modello fisico del pendolo: descrizione di hardware e software}
Per effettuare le esperienze si è utilizzato \textit{Qube-Servo 2}, prodotto dalla Quanser. La piattaforma consiste in un motore CC che si interfaccia con l'ambiente MATLAB/Simulink. Gli schemi di controllo sono dunque creati in Simulink e il motore viene comandato e restituisce dati in tempo reale al computer.
Al motore è possibile connettere un modulo che permette di creare il pendolo inverso (pendolo di Furuta).

In generale, dunque, il sistema prevede:
\begin{description}
    \item [Input: $V_{in}$] La tensione da dare in ingresso al motore. Il motore CC è infatti controllato in tensione.
    \item [Output: $\theta$ e $\alpha$] Derivano dalle misure degli encoder presenti nel sistema: uno permette di ottenere la posizione del motore; l'altro la posizione del pendolo.
\end{description}

\colorbox{yellow}{Altre immagine/robe da scrivere.... bisogna mettere tutte le robe tecniche} 


\begin{figure}[H]
	\centering
	\includegraphics[width=.3\columnwidth]{example-image-a}
	\caption{Evoluzione forzata del sistema in corrispondenza degli ingressi $u_1(t)$ e $u_2(t)$. Si considerano condizioni iniziali nulle}
	\label{fig:1}
\end{figure}









\subsection{Descrizione del problema}




\subsection{Descrizione della soluzione}




\section{Attività Sperimentali}

\subsection{Design del controllore LQR}
\subsection{Dicretizzazione del controllore}
\subsection{Position tracking}
\subsection{Swing-up controller}



\end{document}